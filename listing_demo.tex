% Kurzes Demo zum Gebrauch des listings-Pakets, um Programmcode
% in LaTeX-Dokumente einzubinden.
\documentclass[12pt, ngerman]{scrartcl}

\usepackage[utf8]{inputenc}
\usepackage[T1]{fontenc}
\usepackage{babel}
\usepackage{csquotes}

% Zum Einbinden von Programmcode verwenden wir das listings-Paket
\usepackage{listings}
% textcomp ist für korrekte Anführungsstriche in Code-Abshcnitten
% nötig (Option upquote=true in listings)
\usepackage{textcomp}

% Für Syntax-Highlighting:
\usepackage{xcolor}

% lstset setzt Optionen des listings-Pakets

% Die folgenden listings-Einstellungen sind nötig, um
% deutsche Umlaute und die Tilde (~) in listings-Umgebungen
% verwenden zu können.
\lstset{
  basicstyle=\ttfamily,
  literate={~} {$\sim$}{1} % set tilde as a literal
           {ö}{{\"o}}1
           {ä}{{\"a}}1
           {ü}{{\"u}}1
           {ß}{{\ss}}1
           {Ö}{{\"O}}1
           {Ä}{{\"A}}1
           {Ü}{{\"U}}1
}

% spezieller Stil für bash-Shell:
\lstdefinestyle{bash}{language=bash,%
 commentstyle=\color{red},       % comment style
 keywordstyle=\color{blue},      % keyword style
 upquote=true,     % Korrekte Anführungsstriche in Bash-Codes
 showspaces=false, % Keine speziellen Zeichen für Spaces
 showstringspaces=false,
 breaklines=true,  % lange Zeilen automatisch umbrechen
 postbreak=\mbox{{$\hookrightarrow$}\space}
}

% definiere neue Umgebung für bash-Code:
\lstnewenvironment{lstbash}
{ \lstset{style=bash} } {}

% neues Kommando zum einfach Einbinden von bash-Quelldateien:
\newcommand{\lstinputbash}{\lstinputlisting[style=bash]}

\begin{document}
%
\section{Programmcodes}
Einfacher Code (keine spezifische Programmiersprache) wird mit der
\texttt{lstlistings}-Umgebung eingebunden:
%
\begin{lstlisting}
user$ ls
datei_1.txt  datei_2.txt
\end{lstlisting}
%
Hier \texttt{bash}-spezifischer Code. Wir haben in der Präambel
Syntaxhighlighting und eine extra Umgebung \texttt{lstbash} dafür definiert:
%
\begin{lstbash}
# Hier eine einfache Bash for-Schleife
for NUMMER in 1 2 3 4 5
  touch datei_${NUMMER}.txt
done
echo 'Schleife beendet'
\end{lstbash}
%
Hier ein Beispiel, um Code im Fliesstext einzubinden: Der Befehl
\lstinline{mv *txt copy} verschiebt alle Textdateien in das
Unterverzeichnis \lstinline{copy}.

Zum Schluss noch einen Code-Block, den wir direkt aus einer \textit{Bash}-Datei
nehmen. Hierzu haben wir uns den Befehl \texttt{lstinputbash} definiert:
%
\lstinputbash{for_schleife.sh}

\end{document}
